\documentclass[journal,12pt,twocolumn]{IEEEtran}
%
\usepackage{setspace}
\usepackage{gensymb}
%\doublespacing
\singlespacing

%\usepackage{graphicx}
%\usepackage{amssymb}
%\usepackage{relsize}
\usepackage[cmex10]{amsmath}
\usepackage{siunitx}
%\usepackage{amsthm}
%\interdisplaylinepenalty=2500
%\savesymbol{iint}
%\usepackage{txfonts}
%\restoresymbol{TXF}{iint}
%\usepackage{wasysym}
\usepackage{amsthm}
%\usepackage{iithtlc}
\usepackage{mathrsfs}
\usepackage{txfonts}
\usepackage{stfloats}
\usepackage{steinmetz}
\usepackage{supertabular}
%\usepackage{bm}
\usepackage{cite}
\usepackage{cases}
\usepackage{subfig}
%\usepackage{xtab}
\usepackage{longtable}
\usepackage{multirow}
%\usepackage{algorithm}
%\usepackage{algpseudocode}
\usepackage{enumitem}
\usepackage{mathtools}
\usepackage{tikz}
\usepackage{circuitikz}
\usepackage{verbatim}
\usepackage{tfrupee}
\usepackage[breaklinks=true]{hyperref}
%\usepackage{stmaryrd}
\usepackage{tkz-euclide} % loads  TikZ and tkz-base
%\usetkzobj{all}
\usepackage{pgfplots}
%\usetikzlibrary{all}
\usetikzlibrary{calc,math}
\usetikzlibrary{fadings}
\usetikzlibrary{automata, positioning, arrows}
\usepackage{listings}
    \usepackage{color}                                            %%
    \usepackage{array}                                            %%
    \usepackage{longtable}                                        %%
    \usepackage{calc}                                             %%
    \usepackage{multirow}                                         %%
    \usepackage{hhline}                                           %%
    \usepackage{ifthen}                                           %%
  %optionally (for landscape tables embedded in another document): %%
    \usepackage{lscape}     
\usepackage{multicol}
\usepackage{chngcntr}
\usepackage{blkarray}
%\usepackage{enumerate}

%\usepackage{wasysym}
%\newcounter{MYtempeqncnt}
\DeclareMathOperator*{\Res}{Res}
%\renewcommand{\baselinestretch}{2}
\renewcommand\thesection{\arabic{section}}
\renewcommand\thesubsection{\thesection.\arabic{subsection}}
\renewcommand\thesubsubsection{\thesubsection.\arabic{subsubsection}}

\renewcommand\thesectiondis{\arabic{section}}
\renewcommand\thesubsectiondis{\thesectiondis.\arabic{subsection}}
\renewcommand\thesubsubsectiondis{\thesubsectiondis.\arabic{subsubsection}}

% correct bad hyphenation here
\hyphenation{op-tical net-works semi-conduc-tor}
\def\inputGnumericTable{}                                 %%

\lstset{
%language=C,
frame=single, 
breaklines=true,
columns=fullflexible
}
%\lstset{
%language=tex,
%frame=single, 
%breaklines=true
%}

\begin{document}
%


\newtheorem{theorem}{Theorem}[section]
\newtheorem{problem}{Problem}
\newtheorem{proposition}{Proposition}[section]
\newtheorem{lemma}{Lemma}[section]
\newtheorem{corollary}[theorem]{Corollary}
\newtheorem{example}{Example}[section]
\newtheorem{definition}[problem]{Definition}
%\newtheorem{thm}{Theorem}[section] 
%\newtheorem{defn}[thm]{Definition}
%\newtheorem{algorithm}{Algorithm}[section]
%\newtheorem{cor}{Corollary}
\newcommand{\BEQA}{\begin{eqnarray}}
\newcommand{\EEQA}{\end{eqnarray}}
\newcommand{\define}{\stackrel{\triangle}{=}}

\bibliographystyle{IEEEtran}
%\bibliographystyle{ieeetr}


\providecommand{\mbf}{\mathbf}
\providecommand{\pr}[1]{\ensuremath{\Pr\left(#1\right)}}
\providecommand{\qfunc}[1]{\ensuremath{Q\left(#1\right)}}
\providecommand{\sbrak}[1]{\ensuremath{{}\left[#1\right]}}
\providecommand{\lsbrak}[1]{\ensuremath{{}\left[#1\right.}}
\providecommand{\rsbrak}[1]{\ensuremath{{}\left.#1\right]}}
\providecommand{\brak}[1]{\ensuremath{\left(#1\right)}}
\providecommand{\lbrak}[1]{\ensuremath{\left(#1\right.}}
\providecommand{\rbrak}[1]{\ensuremath{\left.#1\right)}}
\providecommand{\cbrak}[1]{\ensuremath{\left\{#1\right\}}}
\providecommand{\lcbrak}[1]{\ensuremath{\left\{#1\right.}}
\providecommand{\rcbrak}[1]{\ensuremath{\left.#1\right\}}}
\theoremstyle{remark}
\newtheorem{rem}{Remark}
\newcommand{\sgn}{\mathop{\mathrm{sgn}}}
\providecommand{\abs}[1]{\left\vert#1\right\vert}
\providecommand{\res}[1]{\Res\displaylimits_{#1}} 
\providecommand{\norm}[1]{\left\lVert#1\right\rVert}
%\providecommand{\norm}[1]{\lVert#1\rVert}
\providecommand{\mtx}[1]{\mathbf{#1}}
\providecommand{\mean}[1]{E\left[ #1 \right]}
\providecommand{\fourier}{\overset{\mathcal{F}}{ \rightleftharpoons}}
%\providecommand{\hilbert}{\overset{\mathcal{H}}{ \rightleftharpoons}}
\providecommand{\system}{\overset{\mathcal{H}}{ \longleftrightarrow}}
	%\newcommand{\solution}[2]{\textbf{Solution:}{#1}}
\newcommand{\solution}{\noindent \textbf{Solution: }}
\newcommand{\cosec}{\,\text{cosec}\,}
\providecommand{\dec}[2]{\ensuremath{\overset{#1}{\underset{#2}{\gtrless}}}}
\newcommand{\myvec}[1]{\ensuremath{\begin{pmatrix}#1\end{pmatrix}}}
\newcommand{\mydet}[1]{\ensuremath{\begin{vmatrix}#1\end{vmatrix}}}
\newcommand*{\permcomb}[4][0mu]{{{}^{#3}\mkern#1#2_{#4}}}
\newcommand*{\perm}[1][-3mu]{\permcomb[#1]{P}}
\newcommand*{\comb}[1][-1mu]{\permcomb[#1]{C}}
\newcommand{\Int}{\int\limits}

%\numberwithin{equation}{section}
\numberwithin{equation}{subsection}
%\numberwithin{problem}{section}
%\numberwithin{definition}{section}
\makeatletter
\@addtoreset{figure}{problem}
\makeatother

\let\StandardTheFigure\thefigure
\let\vec\mathbf
%\renewcommand{\thefigure}{\theproblem.\arabic{figure}}
\renewcommand{\thefigure}{\theproblem}
%\setlist[enumerate,1]{before=\renewcommand\theequation{\theenumi.\arabic{equation}}
%\counterwithin{equation}{enumi}


%\renewcommand{\theequation}{\arabic{subsection}.\arabic{equation}}

\def\putbox#1#2#3{\makebox[0in][l]{\makebox[#1][l]{}\raisebox{\baselineskip}[0in][0in]{\raisebox{#2}[0in][0in]{#3}}}}
     \def\rightbox#1{\makebox[0in][r]{#1}}
     \def\centbox#1{\makebox[0in]{#1}}
     \def\topbox#1{\raisebox{-\baselineskip}[0in][0in]{#1}}
     \def\midbox#1{\raisebox{-0.5\baselineskip}[0in][0in]{#1}}

\vspace{3cm}

\title{
%	\logo{
Probability
%	}
}
\author{ G V V Sharma$^{*}$% <-this % stops a space
	\thanks{*The author is with the Department
		of Electrical Engineering, Indian Institute of Technology, Hyderabad
		502285 India e-mail:  gadepall@iith.ac.in. All content in this manual is released under GNU GPL.  Free and open source.}
	
}	
%\title{
%	\logo{Matrix Analysis through Octave}{\begin{center}\includegraphics[scale=.24]{tlc}\end{center}}{}{HAMDSP}
%}


% paper title
% can use linebreaks \\ within to get better formatting as desired
%\title{Matrix Analysis through Octave}
%
%
% author names and IEEE memberships
% note positions of commas and nonbreaking spaces ( ~ ) LaTeX will not break
% a structure at a ~ so this keeps an author's name from being broken across
% two lines.
% use \thanks{} to gain access to the first footnote area
% a separate \thanks must be used for each paragraph as LaTeX2e's \thanks
% was not built to handle multiple paragraphs
%

%\author{<-this % stops a space
%\thanks{}}
%}
% note the % following the last \IEEEmembership and also \thanks - 
% these prevent an unwanted space from occurring between the last author name
% and the end of the author line. i.e., if you had this:
% 
% \author{....lastname \thanks{...} \thanks{...} }
%                     ^------------^------------^----Do not want these spaces!
%
% a space would be appended to the last name and could cause every name on that
% line to be shifted left slightly. This is one of those "LaTeX things". For
% instance, "\textbf{A} \textbf{B}" will typeset as "A B" not "AB". To get
% "AB" then you have to do: "\textbf{A}\textbf{B}"
% \thanks is no different in this regard, so shield the last } of each \thanks
% that ends a line with a % and do not let a space in before the next \thanks.
% Spaces after \IEEEmembership other than the last one are OK (and needed) as
% you are supposed to have spaces between the names. For what it is worth,
% this is a minor point as most people would not even notice if the said evil
% space somehow managed to creep in.



% The paper headers
%\markboth{Journal of \LaTeX\ Class Files,~Vol.~6, No.~1, January~2007}%
%{Shell \MakeLowercase{\textit{et al.}}: Bare Demo of IEEEtran.cls for Journals}
% The only time the second header will appear is for the odd numbered pages
% after the title page when using the twoside option.
% 
% *** Note that you probably will NOT want to include the author's ***
% *** name in the headers of peer review papers.                   ***
% You can use \ifCLASSOPTIONpeerreview for conditional compilation here if
% you desire.




% If you want to put a publisher's ID mark on the page you can do it like
% this:
%\IEEEpubid{0000--0000/00\$00.00~\copyright~2007 IEEE}
% Remember, if you use this you must call \IEEEpubidadjcol in the second
% column for its text to clear the IEEEpubid mark.



% make the title area
\maketitle

%\newpage

\tableofcontents

\newpage

\bigskip

\renewcommand{\thefigure}{\theenumi}
\renewcommand{\thetable}{\theenumi}
%\renewcommand{\theequation}{\theenumi}

%\begin{abstract}
%%\boldmath
%In this letter, an algorithm for evaluating the exact analytical bit error rate  (BER)  for the piecewise linear (PL) combiner for  multiple relays is presented. Previous results were available only for upto three relays. The algorithm is unique in the sense that  the actual mathematical expressions, that are prohibitively large, need not be explicitly obtained. The diversity gain due to multiple relays is shown through plots of the analytical BER, well supported by simulations. 
%
%\end{abstract}
% IEEEtran.cls defaults to using nonbold math in the Abstract.
% This preserves the distinction between vectors and scalars. However,
% if the journal you are submitting to favors bold math in the abstract,
% then you can use LaTeX's standard command \boldmath at the very start
% of the abstract to achieve this. Many IEEE journals frown on math
% in the abstract anyway.

% Note that keywords are not normally used for peerreview papers.
%\begin{IEEEkeywords}
%Cooperative diversity, decode and forward, piecewise linear
%\end{IEEEkeywords}



% For peer review papers, you can put extra information on the cover
% page as needed:
% \ifCLASSOPTIONpeerreview
% \begin{center} \bfseries EDICS Category: 3-BBND \end{center}
% \fi
%
% For peerreview papers, this IEEEtran command inserts a page break and
% creates the second title. It will be ignored for other modes.
%\IEEEpeerreviewmaketitle

\begin{abstract}
This book provides solved examples on Probability from IES stats question papers.
\end{abstract}


% \section{June 2019}
% \renewcommand{\theequation}{\theenumi}
\renewcommand{\thefigure}{\theenumi}
\renewcommand{\thetable}{\theenumi}
\begin{enumerate}[label=\thesection.\arabic*.,ref=\thesection.\theenumi]
\numberwithin{equation}{enumi}
\numberwithin{figure}{enumi}
\numberwithin{table}{enumi}

\item Let $X$ be a Poisson random variable with p.m.f
\begin{align}
\label{eq:1}
P(X=k) = 
    \begin{cases} 
      \frac{e^{-\lambda}\lambda^{k}}{k!},& k=0,1,2,...;  \lambda > 0\\
      0 & \text{otherwise}
   \end{cases}
\end{align}
If $Y = X^2 + 3$, then what is $P(Y=y)$ equal to?
\begin{enumerate}[label={(\Alph*)}]
    \item $\frac{e^{-\lambda}\lambda^{\sqrt{y-3}}}{\sqrt{\brak{y-3}}!}$, for $y =$ \cbrak{3,4,7,12,...}
    \item $\frac{e^{-\lambda}\lambda^{-\sqrt{y-3}}}{\sqrt{\brak{3-y}}!}$, for $y =$ \cbrak{3,4,7,12,...}
    \item $\frac{e^{-\lambda}\lambda^{\sqrt{3-y}}}{\sqrt{\brak{3-y}}!}$, for $y =$ \cbrak{4,7,12,...}
    \item $\frac{e^{-\lambda}\lambda^{-\sqrt{3-y}}}{\sqrt{\brak{3-y}}!}$, for $y =$ \cbrak{4,7,12,...}
\end{enumerate}
%
\solution
%\input{solutions/2019/16.tex}
%
\item Two cannons $A_{1}$ and $A_{2}$ fire at the same target. Cannon $A_{1}$ fires on an average 9 projectiles in the time in which cannon $A_{2}$ fire 10 projectiles.But on an average 7 out of 10 projectiles from cannon $A_{1}$ and 6 out of 10 projectiles from cannon $A_{2}$ strike the target. If in the course of shooting, the target is struck by one projectile, then the probability that it is struck by projectile from cannon $A_{1}$ is 
\begin{enumerate}
   \item  $\frac{20}{41}$\\
   \item  $\frac{21}{41}$ \\ 
   \item  $\frac{6}{19}$ \\
   \item  $\frac{63}{190}$ \\ 
\end{enumerate}
%
\solution
\begin{table}[h!]
    \centering
    \caption{Listing of events }
    \label{2019/12/table:1}
    \begin{tabular}{|c|c|}
        \hline
        S & {\small event that the target is struck by a projectile}\\[0.5ex]
        \hline
        $A_{1}$ & {\small event that cannon $A_{1}$ fires a projectile}\\[0.5ex]
        \hline
        $A_{2}$ & {\small event that cannon $A_{2}$ fires a projectile}\\[0.5ex]
        \hline
    \end{tabular}
    \end{table}
    We need to calculate the conditional probability $\pr{A_{1}|S}$.\\
    By Bayes' Theorem, we get 
    \begin{align}
    \label{2019/12/eq:1}
    \pr{A_{1}|S}=\frac{\pr{S|A_{1}}\pr{A_{1}}}{\pr{S|A_{1}}\pr{A_{1}}+\pr{S|A_{2}}\pr{A_{2}}}
    \end{align}
    In \eqref{2019/12/eq:1},\\
     $\pr{S|A_{i}}$ represents the conditional probability of cannon $A_{i}$ striking the target.\\
    Given,
    \begin{table}[h!]
    \centering
    \label{2019/12/table:2}
    \begin{tabular}{|c|c|c|}
        \hline
        E & $A_{1}$ & $A_{2}$\\
        \hline
        $\pr{E}$ & $\frac{9}{19}$ & $\frac{10}{19}$\\[1ex]
        \hline
    \end{tabular}
    \end{table}
    Also,
    \begin{align}
    \pr{S|A_{1}}=\frac{7}{10}\\
    \pr{S|A_{2}}=\frac{6}{10}\\
    \end{align}
    Substituting the values in \eqref{2019/12/eq:1}, we get
    \begin{align}
    \pr{A_{1}|S}={}&\frac{\frac{7}{10}\frac{9}{19}}{\frac{7}{10}\frac{9}{19}+\frac{6}{10}\frac{10}{19}}\\
    ={}&\frac{63}{63+60}\\
    ={}&\frac{21}{41}
    \end{align}
     Therefore, option 2 is correct.
\end{enumerate}

% \section{December 2018}
% \input{./chapters/2018/dec.tex}
% \section{June 2018}
% \input{./chapters/2018/june.tex}
% \section{December 2017}
% \input{./chapters/2017/dec.tex}
% \twocolumn
% \section{June 2017}
% \input{./chapters/2017/june.tex}
% \section{December 2016}
% \input{./chapters/2016/dec.tex}
% \section{June 2016}
% \input{./chapters/2016/june.tex}
\section{ 2015}
 \renewcommand{\theequation}{\theenumi}
\renewcommand{\thefigure}{\theenumi}
\renewcommand{\thetable}{\theenumi}
\begin{enumerate}[label=\thesection.\arabic*.,ref=\thesection.\theenumi]
\numberwithin{equation}{enumi}
\numberwithin{figure}{enumi}
\numberwithin{table}{enumi}
%


\item For random variables X and Y, show that:
$Var[Y] = E[Var(Y|X)] + Var[E(Y|X)]$
%
\solution

Let the abbreviations LE and LIE denote linearity of expectations and law of iterated expectations respectively.
\begin{align}
Var [Y] &= E[Y^2] -[E(Y)]^2\text{  (definition)  }\\
        &= E[E(Y^2 | X )] - (E[E(Y | X)])^2\text{  (LIE)  }
\end{align}
\begin{multline}
    =E[E(Y^2 | X )] - (E[E(Y | X)])^2\\-E([E(Y | X)]^2) + E([E(Y | X)]^2)
\end{multline}
\begin{multline}
        =E[E(Y^2 | X )] - E([E(Y | X)]^2) \\ + E([E(Y | X)]^2) - (E[E(Y | X)])^2\text{  (LE {\&} LIE)  }
\end{multline}
\begin{align}
        &= Var [E(Y | X)] + E[Var (Y | X)]\text{  (definition)  }
\end{align}
\bigskip
\rightline{Hence, proved.}
%
\item Let X be a Random Variable with $E[X] = 3$, $E[X^2] = 13$. Use Chebyshev's Inequality to obtain $\pr{-2 < X < 8}$
%
\\
\solution
Let X be a random variable with finite expected value $E[X]$ and finite non-zero variance $\sigma^2$. Then for any real number k $>$ 0,
\begin{equation*}
\tag{1} \label{ies2015-2:chebyshev}
    \pr{\lvert X - E[X] \rvert \ge k\sigma} \le \frac{1}{k^{2}}
\end{equation*}
computing the variance,
\begin{align*}
    \sigma^{2} &= E[X^2] - E[X]^2 \\
    \tag{2}
    \implies \sigma^{2} &= 13 - 9 = 4 \\
    \tag{3} \label{ies2015-2:sigma}
    \sigma &= 2
\end{align*}
using \eqref{ies2015-2:sigma},
\begin{align*}\
\tag{4} \label{ies2015-2:final}
    \pr{-2 < X < 8} &= 1 - \pr{\lvert X - 3 \rvert > 5} \\
    \tag{5} \label{ies2015-2:main}
    \pr{\lvert X - 3 \rvert > 5} &= \pr{\lvert X - E[X] \rvert > k\sigma} \\
    k\sigma &= 5 \\
    \implies 2k &= 5 \\
    \tag{6} \label{ies2015-2:k}
    \therefore k &= \frac{5}{2}
\end{align*}
Using \eqref{ies2015-2:chebyshev} , \eqref{ies2015-2:main} and \eqref{ies2015-2:k} in \eqref{ies2015-2:final},
\begin{align*}
    \pr{-2 < X < 8} &\ge 1 - \brak{\frac{2}{5}}^2 \\
    \tag{7} \label{ies2015-2:result}
\implies \pr{-2 < X < 8} &\ge \frac{21}{25}
\end{align*}

%
\item  Three points are chosen on the line of unit length.Find the probability that each the 3 line segments have length greater than $\dfrac{1}{4}$.
\\
\solution

  Let $X,Y \in \{0,1\}$ be the random variables which represent the position of two points on the line of unit length.\\
Conditions which should be satisfied to have three line segments with length greater than 
\begin{table}[h]
\centering
\bgroup
\def\arraystretch{2}
\begin{tabular}{|c|c|}
\hline
\textbf{Event} & \textbf{Condition}                     \\\hline
A              & $\dfrac{1}{4}<X<\dfrac{3}{4}$ \\[1ex] \hline
B              & $\dfrac{1}{4}<Y<\dfrac{3}{4}$ \\[1ex] \hline
C              & $\dfrac{1}{4}<X-Y$ \\[1ex] \hline
D           & $\dfrac{1}{4}<Y-X$ \\[1ex] \hline
\end{tabular}
\egroup
\caption{Events and their conditions}
\label{2015/3/tab:Events}
\end{table}
$\frac{1}{4}$ are given in the below table.\\
Then the required event which solves the problem is $ABC$+$ABD$.
\begin{align}
    \pr{ABC}&=\pr{\frac{1}{4}+Y<X,\frac{1}{4}<X,Y<\frac{3}{4}}\\
    &=\sum~\pr{Y=y|\frac{1}{4}<X,Y<\frac{3}{4}}\times\nonumber\\
    &~~~~~~~~~~\pr{\frac{1}{4}+y<X,\frac{1}{4}<X<\frac{3}{4}}\\
    &=\int_{\frac{1}{4}}^{\frac{3}{4}}dyf_Y(y)\times\nonumber\\ &~~~~~~~\pr{\frac{1}{4}+y<X,\frac{1}{4}<X<\frac{3}{4}}\\
     &=\int_{\frac{1}{4}}^{\frac{3}{4}}dyf_Y(y)\pr{\frac{1}{4}+y<X<\frac{3}{4}}\label{2015/3/step2}
     \end{align}
     As $X$ is distributed uniformly between 0 and 1.
     \begin{align}
        \pr{\frac{1}{4}+y<X<\frac{3}{4}}=\begin{cases}
        \dfrac{1}{2}-y&y\in \brak{0,\dfrac{1}{2}}\\
        0&\text{otherwise}
        \end{cases}\label{2015/3/step2help}
     \end{align}
     Using \eqref{2015/3/step2help},\eqref{2015/3/step2} can be written as
     \begin{align}
    \pr{ABC}&=\int_{\frac{1}{4}}^{\frac{1}{2}}dyf_Y(y)\brak{\frac{1}{2}-y}
    \end{align}
    As y is distributed uniformly between 0 and 1.
    \begin{align}
   \pr{ABC} &=\int_{\frac{1}{4}}^{\frac{1}{2}}\frac{1}{2}-y~dy\\
    &=\dfrac{1}{32}
\end{align}
Similarly,we can find,
\begin{align}
    \pr{ABD}&=\dfrac{1}{32}
\end{align}
As $C$ and $D$ are mutually exclusive events.
\begin{align}
    \pr{ABC+ABD}&=\pr{ABC}+\pr{ABD}\\
    &=\dfrac{1}{16} 
\end{align}
$\therefore$ probability that each of the three line segments have length greater than $\dfrac{1}{4}$  is  $\dfrac{1}{16}$.

%
\item Two points are chosen on a line of unit length.Find the probability that each of the 3 line segments will have length greater than $\frac{1}{4}$?
\\
\solution
let us choose points X,Y on a line ,since we are picking points randomly they have a uniform distribution in \brak{0,1}.\\
let us take two random variables X,Y.
This points divide in into segments $X,Y-X,1-X$
From question
\begin{align}
X > \frac{1}{4}\\
Y-X> \frac{1}{4}\\
1-X< 1/4
\end{align}
$\binom{2}{1}\times\pr{X+\frac{1}{4}< Y < \frac{3}{4},\frac{1}{4}<X<\frac{1}{2} }$ is the required answer since
 given random variables  X,Y are interchangable.
\begin{align}
    \pr{X+\frac{1}{4}< Y < \frac{3}{4},\frac{1}{4}<X<\frac{1}{2}}\\= \int_{X=\frac{1}{4}}^{X=\frac{1}{2}}\int_{Y=X+\frac{1}{4}}^{Y=\frac{3}{4}}dY dX
    \\=\int_{X=\frac{1}{4}}^{X=\frac{1}{2}}\brak{\frac{1}{2}-X}dX
    \\=\brak{\frac{X}{2}- X^{2}}\bigg|_{\frac{1}{4}}^{\frac{1}{2}}
    \\=\frac{1}{8}-\frac{1}{8}+\frac{1}{32}
    \\=\frac{1}{32}
\end{align}
So,req answer is $$2\times\pr{X+\frac{1}{4}< Y < \frac{3}{4},\frac{1}{4}<X<\frac{1}{2} }=\frac{1}{16}$$
  \textbf{Ans is} $\frac{1}{16}$

\end{enumerate}

 \section{ 2016}
 \renewcommand{\theequation}{\theenumi}
\renewcommand{\thefigure}{\theenumi}
\renewcommand{\thetable}{\theenumi}
\begin{enumerate}[label=\thesection.\arabic*.,ref=\thesection.\theenumi]
\numberwithin{equation}{enumi}
\numberwithin{figure}{enumi}
\numberwithin{table}{enumi}
%
\item Let the random variable X have the distribution $P(X=0)=P(X=3)=p$, $P(X=1)=1-3p$ for $0{\leq}p{\leq}\frac{1}{2}$. What is the maximum value of V(X)?
\begin{enumerate}[label=\Alph*)]
    \item 3
    \item 4
    \item 5
    \item 6
    \item none
\end{enumerate}
%
\solution
Given, for $0{\leq}p{\leq}\frac{1}{2}$,
\begin{align}
    P(X=0)&=p\\
    P(X=1)&=1-3p\\
    P(X=3)&=p
\end{align}
Now consider $P(X=1)=1-3p$ for $p=\frac{1}{2}$. We get,
\begin{align}
    P(X=1)&=1-3p\\
    &=1-(3)\brak{\frac{1}{2}}\\
    &=1-{\frac{3}{2}}\\
    &=-\frac{1}{2}<0
\end{align}
Probability cannot be negative. But in equation (0.0.7) probability is negative, which is not possible.\\
Therefore, the question is not a proper one.\\
\rightline{Answer : Option E}
%
\item $X_1$ and $X_2$ are independent Poisson variables such that $\pr{X_1=2} = \pr{X_1=1}$ and $\pr{X_2=2} = \pr{X_2=3}$. What is the variance of $(X_1 - 2X_2)$ ?
\begin{enumerate}
    \item 14
    \item 4
    \item 3
    \item 2
\end{enumerate}
%
\solution
\newcommand{\E}{\mathrm{E}}
\newcommand{\Var}{\mathrm{Var}}
For a Poisson variable X,
\begin{align}
\pr{X=k} = \frac{\lambda^{k}e^{-\lambda}}{k!}
\end{align}
Since $\pr{X_1=2} = \pr{X_1=1}$,
\begin{align}
\frac{{\lambda_1}^{2}e^{-{\lambda_1}}}{2!} &= \frac{{\lambda_1}^{1}e^{-{\lambda_1}}}{1!} \\
\lambda_1 &= 2!/1! = 2
\end{align}
Similarly, as $\pr{X_2=2} = \pr{X_2=3}$,
\begin{align}
\frac{{\lambda_2}^{2}e^{-{\lambda_2}}}{2!} &= \frac{{\lambda_2}^{3}e^{-{\lambda_2}}}{3!} \\
\lambda_2 &= 3!/2! = 3
\end{align}
Also we know for a Poisson variable X, the following holds true:
\begin{align}
\E[X] &= \lambda \\
\Var[X] &= \lambda \label{2016/4eq1} \\
\Var[X] &= \E[X^2] - (\E[X])^2 \label{2016/4eq2} 
\end{align}
Now, for the variance of $(X_1 - 2X_2)$
\begin{align}
\Var[X_1 - 2X_2] &= \E[(X_1 - 2X_2)^2] - (\E[X_1 - 2X_2])^2 \nonumber \\
&= \E[X_1^2 + 4X_2^2 - 4X_1X_2] \nonumber \\
&- (\E[X_1] - 2\E[X_2])^2 \nonumber \\
&= \E[X_1^2] - (\E[X_1])^2 + 4\E[X_2^2] \nonumber \\
& -4(\E[X_2])^2) + 4\E[X_1X_2] \nonumber \\
& + 4\E[X_1]\E[X_2]
\end{align}
Since the variables are independent:
\begin{align}
\E[X_1X_2] = \E[X_1]\E[X_2]
\end{align}
Substituting equations \eqref{2016/4eq1} and \eqref{2016/4eq2}, we get:
\begin{align}
\Var[X_1 - 2X_2] &= \Var[X_1] + 4(\Var[X_2]) \nonumber \\
&- 4\E[X_1][X_2] + 4\E[X_1][X_2] \nonumber \\
&= \lambda_1 + 4\lambda_2 = 2 + 4(3) = 14
\end{align}
Hence option (a) 14 is correct.

\end{enumerate}



% \section{June 2015}
% \input{./chapters/2015/june.tex}
% \section{December 2014}
% \input{./chapters/2014/dec.tex}
% \section{June 2013}
% \input{./chapters/2013/june.tex}

% \section{December 2012}
% \input{./chapters/2012/dec.tex}



\end{document}


